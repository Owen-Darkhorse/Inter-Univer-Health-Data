% Options for packages loaded elsewhere
\PassOptionsToPackage{unicode}{hyperref}
\PassOptionsToPackage{hyphens}{url}
%
\documentclass[
]{article}
\usepackage{amsmath,amssymb}
\usepackage{iftex}
\ifPDFTeX
  \usepackage[T1]{fontenc}
  \usepackage[utf8]{inputenc}
  \usepackage{textcomp} % provide euro and other symbols
\else % if luatex or xetex
  \usepackage{unicode-math} % this also loads fontspec
  \defaultfontfeatures{Scale=MatchLowercase}
  \defaultfontfeatures[\rmfamily]{Ligatures=TeX,Scale=1}
\fi
\usepackage{lmodern}
\ifPDFTeX\else
  % xetex/luatex font selection
\fi
% Use upquote if available, for straight quotes in verbatim environments
\IfFileExists{upquote.sty}{\usepackage{upquote}}{}
\IfFileExists{microtype.sty}{% use microtype if available
  \usepackage[]{microtype}
  \UseMicrotypeSet[protrusion]{basicmath} % disable protrusion for tt fonts
}{}
\makeatletter
\@ifundefined{KOMAClassName}{% if non-KOMA class
  \IfFileExists{parskip.sty}{%
    \usepackage{parskip}
  }{% else
    \setlength{\parindent}{0pt}
    \setlength{\parskip}{6pt plus 2pt minus 1pt}}
}{% if KOMA class
  \KOMAoptions{parskip=half}}
\makeatother
\usepackage{xcolor}
\usepackage[margin=1in]{geometry}
\usepackage{graphicx}
\makeatletter
\newsavebox\pandoc@box
\newcommand*\pandocbounded[1]{% scales image to fit in text height/width
  \sbox\pandoc@box{#1}%
  \Gscale@div\@tempa{\textheight}{\dimexpr\ht\pandoc@box+\dp\pandoc@box\relax}%
  \Gscale@div\@tempb{\linewidth}{\wd\pandoc@box}%
  \ifdim\@tempb\p@<\@tempa\p@\let\@tempa\@tempb\fi% select the smaller of both
  \ifdim\@tempa\p@<\p@\scalebox{\@tempa}{\usebox\pandoc@box}%
  \else\usebox{\pandoc@box}%
  \fi%
}
% Set default figure placement to htbp
\def\fps@figure{htbp}
\makeatother
\setlength{\emergencystretch}{3em} % prevent overfull lines
\providecommand{\tightlist}{%
  \setlength{\itemsep}{0pt}\setlength{\parskip}{0pt}}
\setcounter{secnumdepth}{-\maxdimen} % remove section numbering
\usepackage{bookmark}
\IfFileExists{xurl.sty}{\usepackage{xurl}}{} % add URL line breaks if available
\urlstyle{same}
\hypersetup{
  pdftitle={Statistical Method},
  pdfauthor={250539-PCA-Next-Step},
  hidelinks,
  pdfcreator={LaTeX via pandoc}}

\title{Statistical Method}
\author{250539-PCA-Next-Step}
\date{2025-06-25}

\begin{document}
\maketitle

During the exploratory analysis stage, plot the median and mean
cognitive scores of a subset of the sample against age and occupation
provides intuitions of factors of cognitive decline. As the goal is to
identify occupations that alleviate cognitive decline the most,
comparing the effect of occupations against retirement over time and
determining the statistical significance are necessary. Therefore, we
will build a statistical model for the longitudinal data to quantify
effects and test hypotheses whether certain occupations effectively
preserves cognitive abilities.

\subsubsection{Generalized Estimating Equation (GEE)}

We can assume that reference persons from different households are
independent, but each subject were measured for their recall, mental
status, and cognitive impairment indicator across waves. These repeated
measurements within subjects are correlated. Additionally, response
variables are integer valued; in particular, the distribution of mental
status is left skewed (figure \ldots), while the cognitive impairment
indicator is binary valued. This implies that the ordinary least squares
(OLS) no longer apply as normality and independence assumptions are
violated. Furthermore, (Figure \ldots) shows that recall and mental
state are non-linearly correlated with age as the rate of decline
accelerates after 70 years of age. As the true data-generating process
is unknown, model misspecification could cause invalid inference. These
issues motivate the application of GEE.

Let \(Y_{ij}\) be the response variable (i.e, recall and mental scores)
of i-th subject at the j-th measurement, where \(1 \leq i \leq n\), and
\(1 \leq j \leq k_i\). And let \(\bf x_{ij}\) be the independent
variable of the i-th subject at the j-th measurement. To model the
marginal mean \(\mu_{ij} = E(Y_{ij})\), we impose a few assumptions:

\begin{enumerate}
    \item $Y_{ij} \sim Binomial(m, p_{ij})$ is the random component of the model, where $m=20$ for recalls and $m=15$ for mental score, and $p_{ij}$ is the unknown probability of gaining 1 more point in the exam.
    \item The systematic or linear component is $\eta_{ij} = \mathbf{x_{ij}}^T \beta$, where $\beta$ is a vector of unkown parameters,representing the effect contributed by each variable.
    \item The link function $\eta_{ij} = g(\mu_{ij}) = \mathbf{x_{ij}}^T \beta$ maps the expectations of $Y_{ij}$ between 0 and 20 to a set of real numbers, a linear regression will be fitted against the transformed response.
    \item $Var(Y_i) = a(\phi) V(\mu_i)$ captures the covariance structure within the i-th subject. $a(\phi)$ is a function of scaling parameter $\phi$ parameter $\phi$, and $V(\mu_i)$ is variance function with respect to the mean vector $\mu_i$.
\end{enumerate}

By assumptions above, the distribution function of \(Y_{ij}\) is:

\[
f(y_{ij}; p_{ij}) = {m \choose y_{ij}} p_{ij}^{p_{ij}}  (1-p_{ij})^{m-y_{ij}}
\] To be written in the general form of exponential family is

\$\$ f(y\_\{ij\}; p\_\{ij\}) = \exp\{
\frac{y \theta - b(\theta)}{a(\phi)} + c(y; \phi)\}

=\exp\{ylog(\frac{p_{ij}}{1-p_{ij}}) + mlog(1- p\_\{ij\}) + log\{m
\choose y\_\{ij\}\} \} \$\$ Denote the
\(\theta_{ij} = log(\frac{p_{ij}}{1-p_{ij}})\) (the logit function) to
be the canonical parameter, then
\(p_{ij} = expit(\theta_{ij}) = \frac{e^{\theta_{ij}}}{1 + e^{\theta_{ij}}}\).
Matching the rest of parts, \(a(\phi) = 1\), and
\(b(\theta) = -mlog(1- p_{ij}) = m*log(1 + e^{\theta_{ij}})\),
\(c(y_{ij}; \phi) = log{m \choose y_{ij}}\).

\end{document}
